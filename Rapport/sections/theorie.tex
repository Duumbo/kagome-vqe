\documentclass[../main.tex]{subfiles}

\begin{document}

  \section{Théorie}
  L'estimation de l'état fondamental d'un réseau appartient aux problèmes à $N$
  corps. Ce type de problème est très difficile à résoudre en général. Pour cette
  raison, le réseau utilisé dans ce projet comportera $12$ sites afin de
  restreindre la complexité du calcul. L'état fondamental sera estimé en utilisant
  l'hamiltonien de Heisenberg, soit
  \begin{align}
      H&=t\sum_{<i,j>}\mathbf{S}_i\cdot\mathbf{S}_j
  \end{align}
  avec $t$ une constante de couplage positive, et $\qty(\mathbf{S}_i)_j=\sigma_j,\ j\in\{x,y,z\}$,
  l'opérateur de moment cinétique intrinsèque au site $i$.
  Cet hamiltonien permet de capturer
  le phénomène d'anti-ferromagnétisme en priviliégeant les état pour lesquels
  les spins sont anti-alignés. L'état fondamental de cet hamiltonien sur le réseau
  kagome ne sera pas simple, car le réseau est frustré géométriquement, donnant
  une grande dégénérescence.
  \subsection{Diagonalisation exacte} % (fold)
  \label{sub:Diagonalisation exacte}
    La diagonalisation exacte, malgré son nom, permet une approximation de l'état
    fondamental en diagonalisant l'hamiltonien, la matrice décrivant la dynamique
    du système, ayant pour valeurs propres les énergies possible du système.
    En général, les hamiltoniens qu'on peut écrire pour les problèmes
    à $N$ corps
    sont de dimension exponentielle en terme du nombre de site,
    alors l'écriture de
    l'hamiltonien se fait dans un plus petit sous-espace. Même pour $12$ sites,
    cet hamiltonien peut être trop grand pour pouvoir faire des manipualtions
    dans un temps raisonnable. Pour cette raison, l'encodage utilisé fera usage
    du fait que le modèle d'Heisenberg utilise des matrices de Pauli, ce qui
    permet l'usage de chaines de Pauli et de matrices creuses. L'utilisation
    de matrices creuses est pertinente, car elles permettent d'utiliser des matrices
    de grande dimension sans prendre toute la mémoire d'une matrice explicite.
    Ceci permet d'utiliser la méthode de Lanczos pour déterminer la valeur
    propre minimale, soit l'énergie minimale\cite{senech}. La méthode de Lanczos repose sur
    le fait qu'on peut restreindre l'espace des vecteurs propres au sous-espace
    de Krylov, dans lequel on peut écrire une matrice tridiagonale de dimension
    beaucoup plus petite dont les valeurs propres extrèmes convergent aux mêmes
    que la matrice creuse initiale\cite{livreboboche}. Pour ce faire, il est seulement requis de
    savoir comment appliquer la matrice sur un vecteur aléatoire $\ket{\psi_0}$,
    qui permet ensuite de bâtir une base orthogonale dans laquelle on peut écrire
    la matrice tridiagonale, soit
    \begin{align}
        K&=\span{\ket{\psi_0},H\ket{\psi_0},...,H^{M-1}\ket{\psi_0}}
    \end{align}
    On peut ensuite obtenir une base du sous-espace de Krylov avec la relation
    de récurence
    \begin{align}
        \ket{\psi_{n+1}}=H\ket{\psi_n}-a_n\ket{\psi_n}-b^2_n\ket{\psi_{n-1}}\\
        a_n=\frac{\expval{\psi_n|H|\psi_n}}{\braket{\psi_n}{\psi_n}}\qquad
        b^2_n=\frac{\braket{\psi_n}{\psi_n}}{\braket{\psi_{n-1}}{\psi_{n-1}}}
    \end{align}
    Ce qui permet ensuite d'écrire la matrice tridiagonale de dimension $M$ qui
    possède les mêmes valeurs propres extrêmes
    \begin{align}
        T&=\begin{pmatrix}
            a_0&b_1&0&\cdots&0\\
            b_1&a_1&b_2&\cdots&0\\
            \vdots&\vdots&\vdots&\ddots&\vdots\\
            0&0&0&\cdots&a_{M-1}
        \end{pmatrix}
    \end{align}
    Des modules déjà
    produits permettent d'implémenter directement ces techniques, tel que
    \textit{QuSpin}\cite{10.21468/SciPostPhys.7.2.020}

  % subsection subsection name (end)

    \subsection{Solutionneur de valeurs propres quantique variationnel} % (fold)
    \label{sub:vqe}
    Le solutionneur de valeurs propres quantique (\textit{Variationnal Quantum
    Eigensolver}) est une technique de diagonalisation qui utilise le
    principe variationnel pour donner une borne supérieure sur l'énergie de
    l'état fondamental. Cette technique dérive de la méthode variationnelle en
    mécanique quantique, soit reposant sur le fait qu'un état propre de
    l'hamiltonien
    \begin{align}
        H\ket{\psi_\lambda}&=E_\lambda\ket{\psi_\lambda}
    \end{align}
    permet d'écrire un état général du système comme une combinaison linéaire.
    \begin{align}
        \ket{\psi}&=\sum_\lambda\alpha_\lambda\ket{\psi_\lambda}
    \end{align}
    Ce qui permet d'écrire la valeur moyenne de $H$ dans l'état $\ket{\psi}$
    \begin{align}
        \bra{\psi}H\ket{\psi}&=\sum_{\lambda,\rho}\qty(\alpha_\rho^*\alpha_\lambda
        \bra{\psi_\rho}H\ket{\psi_\lambda})\\
        &=\sum_\lambda\abs{\alpha_\lambda}^2E_\lambda\geq E_0
    \end{align}
    étant donné que $\sum_\lambda\abs{\alpha_\lambda}^2=1$ et que $E_\lambda\geq
    E_0$. On peut donc utiliser ce théorème pour donner une bonne borne supérieure
    à l'énergie du fondamental en introduisant des paramètres à l'état d'essai
    $\ket{\psi(\{\theta_i\})}$ et en minimisant ainsi la fonction d'énergie
    $E(\{\theta_i\})$. Sur un ordinateur quantique, préparer un tel état est
    faisable en appliquant une porte unitaire $U(\{\theta_i\})$ qui assure un
    bon contrôle sur l'état. Une fois l'état préparé, la valeur moyenne est
    simple à prendre en effectuant les mesures appropriées.

    % subsection subsection name (end)

\end{document}

