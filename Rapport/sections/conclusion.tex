\documentclass[../main.tex]{subfiles}

\begin{document}

  \section*{Conclusion}
  L'énergie de l'état fondamental du modèle de Heisenberg sur un réseau kagome
  a pu être calculée en utilisant la diagonalisation exacte d'un amas de $12$ sites.
  Elle a aussi pu être approximée avec une erreur de $30\%$ en utilisant la
  VQE, un algorithme variationnel quantique pour estimer la valeur propre minimale
  d'un hamiltonien. Cette estimation a été fait en simulation classique d'un
  ordinateur quantique non bruité. Plusieurs état d'essai ont pu être testé,
  celui donnant la meilleure énergie minimale étant celui qui a été bâti sur
  des arguments physiques du système. Il serait intéressant d'envoyer ces
  circuits sur de véritables ordinateurs quantique, or pour avoir un résultat
  qui ait du sens, il faudrait travailler à diminuer la profondeur du circuit,
  diminuer le taux d'erreur et implémenter des techniques de mitigation d'erreur.

\end{document}


