\documentclass[../main.tex]{subfiles}

\begin{document}

  \section*{Introduction}
  Un liquide de spin quantique est une phase de la matière qui découle de la frustration
  dans un matériel. Cette approche pour expliquer le comportement de la matière
  a été apportée par le physicien Philip Warren Anderson en 1973 pour expliquer
  le comportement d'un réseau triangulaire\cite{ANDERSON1973153}
  Les liquides de spins sont intéressant à étudier car il s'agit de systèmes
  qui présentent beaucoup de fluctuations quantiques, ce qui fait d'eux un des
  matériaux avec des propriétés hors de l'ordinaire.
  Un autre type de réseau qui
  présente comportement en liquide de spin similaire est le réseau Kagome. Dans
  ce projet, on cherche à estimer l'état fondamental d'un tel réseau en utilisant
  deux techniques, soit une diagonalisation exacte de l'hamiltonien de
  Heisenberg, et en utilisant une technique variationnelle pour estimer l'état
  fondamental.

\end{document}

